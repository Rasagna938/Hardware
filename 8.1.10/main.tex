\let\negmedspace\undefined
\let\negthickspace\undefined
\documentclass[journal]{IEEEtran}
\usepackage[a5paper, margin=10mm, onecolumn]{geometry}
%\usepackage{lmodern} % Ensure lmodern is loaded for pdflatex
\usepackage{tfrupee} % Include tfrupee package

\setlength{\headheight}{1cm} % Set the height of the header box
\setlength{\headsep}{0mm}     % Set the distance between the header box and the top of the text

\usepackage{gvv-book}
\usepackage{gvv}
\usepackage{cite}
\usepackage{amsmath,amssymb,amsfonts,amsthm}
\usepackage{algorithmic}
\usepackage{graphicx}
\usepackage{textcomp}
\usepackage{xcolor}
\usepackage{txfonts}
\usepackage{listings}
\usepackage{enumitem}
\usepackage{mathtools}
\usepackage{gensymb}
\usepackage{comment}
\usepackage[breaklinks=true]{hyperref}
\usepackage{tkz-euclide} 
\usepackage{listings}
% \usepackage{gvv}                                        
\def\inputGnumericTable{}                                 
\usepackage[latin1]{inputenc}                                
\usepackage{color}                                            
\usepackage{array}                                            
\usepackage{longtable}                                       
\usepackage{calc}                                             
\usepackage{multirow}                                         
\usepackage{hhline}                                           
\usepackage{ifthen}                                           
\usepackage{lscape}
\begin{document}

\bibliographystyle{IEEEtran}
\vspace{3cm}

\title{NCERT-8.1.10}
\author{EE24BTECH11023 - RASAGNA}

% \maketitle
% \newpage
% \bigskip
{\let\newpage\relax\maketitle}

\renewcommand{\thefigure}{\theenumi}
\renewcommand{\thetable}{\theenumi}
\setlength{\intextsep}{10pt} % Space between text and floats


\numberwithin{equation}{enumi}
\numberwithin{figure}{enumi}
\renewcommand{\thetable}{\theenumi}
\textbf{Question:}

Find the are bound by the curve $x^2=4y$ and the line $x=4y-2$.


\textbf{Solution:}
The area bounded by two curves $f(x)$ and $g(x)$ over a specific interval $[a, b]$ is given by:
\begin{align}
A=\int_a^b|f(x) - g(x)| dx.
\end{align}

For the given problem:
\begin{align}
l(x)=\frac{x+2}{4} \text{(line)}
\end{align}
\begin{align}
    c(x)= \frac{x^2}{4} \text{(parabola)}
\end{align}

The points of intersection of the curves are found by solving:
\begin{align}
\frac{x^2}{4}= \frac{x + 2}{4}.
\end{align}

Simplifying:
\begin{align}
x^2 - x - 2 = 0
\end{align}
\begin{align}
    (x - 2)(x + 1)= 0
\end{align}

Thus, the points of intersection are $x = -1$ and $x = 2$, so the interval is $[-1, 2]$.

The area is then:
\begin{align}
A &= \int_{-1}^2 \left| \frac{x + 2}{4} - \frac{x^2}{4} \right| \, dx \\
&= \frac{1}{4} \int_{-1}^2 \left( x + 2 - x^2 \right) \, dx.
\end{align}

Expanding and integrating:
\begin{align}
A &= \frac{1}{4} \left[ \frac{x^2}{2} + 2x - \frac{x^3}{3} \right]_{-1}^2.
\end{align}

Substituting the limits:
\begin{align}
A &= \frac{1}{4} \bigg[ \bigg( \frac{2^2}{2} + 2(2) - \frac{2^3}{3} \bigg) - \bigg( \frac{(-1)^2}{2} + 2(-1) - \frac{(-1)^3}{3} \bigg) \bigg] \\
&= \frac{1}{4} \bigg[ \bigg( 2 + 4 - \frac{8}{3} \bigg) - \bigg( \frac{1}{2} - 2 + \frac{1}{3} \bigg) \bigg]\\
&=\frac{9}{8}
\end{align}
Thus, the area bounded by the curves $x^2 = 4y$ and $x = 4y - 2$ is:
\begin{align}
A = 1.125 \, \text{sq. units}.
\end{align}
\textbf{Numerical Integration Using the Trapezoidal Rule(simulation):}\\
Trapezoidal Rule is evaluates the area under the curves by dividing the area under the curve into smaller trapezoids instead of rectangles.This integration method approximates the region under a function's graph as a trapezoid and computes its area.\\
The final integral;
\begin{align}
{\frac{1}{4}}\int_{-1}^{2}( x+2-x^2)\, dx
\end{align}
let us assume $f(x)=(x+2-x^2)$.\\
\textbf{THE TRAPEZOIDAL RULE}
The trapezoidal rule is a numerical method used to approximate the value of a definite integral. It works by approximating the region under the curve as a series of trapezoids and calculating their areas. \\
To implement this, we discretize the range of $x$-coordinates with a uniform step size $h \to 0$, such that the points are $x_0, x_1, x_2, \dots, x_n$ and $x_{n+1} = x_n + h$. \\
Let $A_n$ represent the sum of all trapezoidal areas up to $x_n$. Given $f(x) =\frac{1}{4} {( x+2-x^2)}$, the difference equation can then be expressed as \\
\begin{align}
    A_n &= \frac{h}{2} \left( f(x_0) + f(x_1) \right) + \frac{h}{2} \left( f(x_1) + f(x_2) \right) + \dots + \frac{h}{2} \left( f(x_{n-1}) + f(x_n) \right)
\end{align}
\begin{align}
    A_{n+1} &= A_n + \frac{h}{2} \left( f(x_{n+1}) + f(x_n) \right)
\end{align}
\begin{align}
    A_{n+1} &= A_n + \frac{h}{2} \left( \left( x_{n+1} + 2 - x_{n+1}^2 \right) + \left( x_n + 2 - x_n^2 \right) \right)
\end{align}
The obtained theoritical solution is 1.125 sq.units.\\
The computational solution is 1.1249988750000004sq.units.


\end{document}